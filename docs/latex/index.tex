
\begin{DoxyPre}
Курсовая работа
НИУ ВШЭ Нижегородский филиал
Факультет: Бизнес-информатики и прикладной математики
Направление: Программная инженерия
Курс: 3
Студент: Канафеев Дамир
\end{DoxyPre}
 Краткое описание писание проекта\+:

Эта программа (далее модуль) является частью программного комплекса, реализующего автоматическое движение автомобиля по полосе движения. Целью этого модуля является корректная обработка входных данных и получение необходимого результата, а именно угла поворота колес автомобиля, его скорость и состояние других устройств автомобиля. Основным средством обработки данных является библиотека компьютерного зрения Open\+C\+V, включающая в себя все основные алгоритмы для решения поставленной задачи. Данный модуль структурно можно представить из совокупности блоков, каждый из которых решает свою задачу. В данной версии программного комплекса можно выделить следующие блоки\+: 
\begin{DoxyPre}
1) Блок Контроллер. Главный блок, который управляет всеми остальными блоками.
2) Блок обмена данными. Этот блок отвечает за обмен данными между обрабатывающим модулем и управляющим модулем.
3) Блок детектора полосы. Непосредственно относится к решению поставленной задачи, а именно поиску направления движения и скорости автомобиля.
\end{DoxyPre}
 В следующих версиях программного комплекса будут разработаны блоки\+: 
\begin{DoxyPre}
1) Блок распознавания знаков. Отвечает за поиск знаков на изображении и распознавании их значения для внесения изменений в движение автомобиля
2) Блок приоритетов. Этот блок отвечает за определение приоритетов движения всех объектов дорожного движения.
3) Блок навигатора. Этот блок отвечает за определение маршрута движения автомобиля.
\end{DoxyPre}
 Деление этого модуля на блоки с заранее определенными характеристиками делает данный модуль очень гибким и дает возможность менять структуру модуля без последствий при соблюдении всех характеристик заменяемого блока. 